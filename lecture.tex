\write18{epstopf \epsfilename}
\documentclass[12pt]{minimal}
\usepackage{lingmacros}
\usepackage{tree-dvips}
\usepackage{amsmath}
\usepackage{blindtext}
\begin{document} 

\textbf{scalar and matrix operation} \\ 

$
5 +
\begin{bmatrix}
    1\ 2\\
    3\ 4\\
\end{bmatrix} = 
\begin{bmatrix}
    6\ 7\\
    8\ 9\\
\end{bmatrix} 
$ \\ 

$
a + \textbf{\textit{B}} = \textbf{\textit{C}} 
$ \\ \\ 

$
5 -
\begin{bmatrix}
    1\ 2\\
    3\ 4\\
\end{bmatrix} = 
\begin{bmatrix}
    4\ 3\\
    2\ 1\\
\end{bmatrix} 
$ \\ 

$
a - \textbf{\textit{B}} = \textbf{\textit{C}}
$ \\ \\ 


$
5 \cdot
\begin{bmatrix}
    1\ 2\\
    3\ 4\\
\end{bmatrix} = 
\begin{bmatrix}
    5\ 10\\
    15\ 20\\
\end{bmatrix}
$ \\ 

$
a \cdot \textbf{\textit{B}} = \textbf{\textit{C}}
$ \\ \\ 


$
12 \div
\begin{bmatrix}
    1\ 2\\
    3\ 4\\
\end{bmatrix} = 
\begin{bmatrix}
    12\ 6\\
    4\ 3\\
\end{bmatrix}
$ \\ 


$
a \div \textbf{\textit{B}} = \textbf{\textit{C}} 
$ \\ \\ 

$
\textbf{matrix and matrix operation} 
$ \\ 

$
\begin{bmatrix}
    1\ 1\\
    2\ 2\\
\end{bmatrix} +
\begin{bmatrix}
    1\ 2\\
    3\ 4\\
\end{bmatrix} =
\begin{bmatrix}
    2\ 3\\
    5\ 6\\
\end{bmatrix}
$ \\

\textbf{\textit{A}}+\textbf{\textit{B}}=\textbf{\textit{C}} \\ \\ 

$
\begin{bmatrix}
    1\ 2\\
    3\ 4\\
\end{bmatrix} -
\begin{bmatrix}
    1\ 1\\
    2\ 2\\
\end{bmatrix} =
\begin{bmatrix}
    0\ 1\\
    1\ 2\\
\end{bmatrix}
$ \\

\textbf{\textit{A}}-\textbf{\textit{B}}=\textbf{\textit{C}} \\ \\ 

$
\begin{bmatrix}
    1\ 1\\
    2\ 2\\
\end{bmatrix}
\begin{bmatrix}
    1\ 2\\
    3\ 4\\
\end{bmatrix} = 
\begin{bmatrix}
    4\ 6\\
    8\ 12\\
\end{bmatrix}
$ \\ 

$
\textbf{\textit{A}}\textbf{\textit{B}} = \textbf{\textit{C}} 
$ \\ \\ 

$
\begin{bmatrix}
    1\ 1\\
    2\ 2\\
\end{bmatrix}
\begin{bmatrix}
    1\ 1\\
    2\ 2\\
\end{bmatrix}^{-1} = 
\begin{bmatrix}
    1\ 0\\
    0\ 1\\
\end{bmatrix}
$ \\ 

$
\textbf{\textit{A}}\textbf{\textit{A}}^{-1} = \textbf{\textit{E}}
$




$
\pagebreak
$



$
\textbf{Hadamard product} 
$ \\ \\ 

$
\begin{bmatrix}
    1\ 1\\
    2\ 2\\
\end{bmatrix}
\odot
\begin{bmatrix}
    1\ 2\\
    3\ 4\\
\end{bmatrix} = 
\begin{bmatrix}
    1\ 2\\
    6\ 8\\
\end{bmatrix}
$ \\


$
\textbf{\textit{A}}\odot\textbf{\textit{B}} = \textbf{\textit{C}} 
$ \\ \\ 

$
\textbf{Dot product} 
$ \\

when $\textbf{\textit{a}} = 
\begin{bmatrix}
    1\\ 2\\
\end{bmatrix}$ and 
$\textbf{\textit{b}} = 
\begin{bmatrix}
    3\\ 4\\
\end{bmatrix} 
$ \\ \\ 

$
\textbf{\textit{a}} \cdot \textbf{\textit{b}} = 
\textbf{\textit{a}}^T\textbf{\textit{b}} = 
\begin{bmatrix}
    1\ 2\\
\end{bmatrix}
\begin{bmatrix}
    3\\
    4\\
\end{bmatrix} = 11
$ \\ \\ 



\end{document}
