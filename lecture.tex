\write18{epstopf \epsfilename}
\documentclass[12pt]{minimal}
\usepackage{lingmacros}
\usepackage{tree-dvips}
\usepackage{amsmath}
\usepackage{blindtext}
\begin{document} 

\textbf{scalar and matrix operation} \\ 

\begin{equation}
    \begin{split}
        5 +
        \begin{bmatrix}
            1\ 2\\
            3\ 4\\
        \end{bmatrix} = 
        \begin{bmatrix}
            6\ 7\\
            8\ 9\\
        \end{bmatrix} \\ \\ 
        a + \textbf{\textit{B}} = \textbf{\textit{C}} 
    \end{split}
\end{equation}


\begin{equation}
    \begin{split}
        5 -
        \begin{bmatrix}
            1\ 2\\
            3\ 4\\
        \end{bmatrix} = 
        \begin{bmatrix}
            4\ 3\\
            2\ 1\\
        \end{bmatrix} \\ \\
        a - \textbf{\textit{B}} = \textbf{\textit{C}}
    \end{split}
\end{equation}



\begin{equation}
    \begin{split}
        5 \cdot
        \begin{bmatrix}
            1\ 2\\
            3\ 4\\
        \end{bmatrix} = 
        \begin{bmatrix}
            5\ 10\\
            15\ 20\\
        \end{bmatrix} \\ \\
        a \cdot \textbf{\textit{B}} = \textbf{\textit{C}}
    \end{split}
\end{equation}


\begin{equation}
    \begin{split}
        12 \div
        \begin{bmatrix}
            1\ 2\\
            3\ 4\\
        \end{bmatrix} = 
        \begin{bmatrix}
            12\ 6\\
            4\ 3\\
        \end{bmatrix} \\ \\
        a \div \textbf{\textit{B}} = \textbf{\textit{C}} 
    \end{split}
\end{equation} \\ \\



$\textbf{matrix and matrix operation}$ \\ 

\begin{equation}
    \begin{split}
        \begin{bmatrix}
            1\ 1\\
            2\ 2\\
        \end{bmatrix} +
        \begin{bmatrix}
            1\ 2\\
            3\ 4\\
        \end{bmatrix} =
        \begin{bmatrix}
            2\ 3\\
            5\ 6\\
        \end{bmatrix} \\ \\ 
        \textbf{\textit{A}}+\textbf{\textit{B}}=\textbf{\textit{C}} 
    \end{split}
\end{equation} \\ \\




\begin{equation}
    \begin{split}
        \begin{bmatrix}
            1\ 2\\
            3\ 4\\
        \end{bmatrix} -
        \begin{bmatrix}
            1\ 1\\
            2\ 2\\
        \end{bmatrix} =
        \begin{bmatrix}
            0\ 1\\
            1\ 2\\
        \end{bmatrix} \\ \\
        \textbf{\textit{A}}-\textbf{\textit{B}}=\textbf{\textit{C}} 
    \end{split} 
\end{equation} \\ \\ 


\begin{equation}
    \begin{split}
        \begin{bmatrix}
            1\ 1\\
            2\ 2\\
        \end{bmatrix}
        \begin{bmatrix}
            1\ 2\\
            3\ 4\\
        \end{bmatrix} = 
        \begin{bmatrix}
            4\ 6\\
            8\ 12\\
        \end{bmatrix} \\ \\
        \textbf{\textit{A}}\textbf{\textit{B}} = \textbf{\textit{C}} 
    \end{split}
\end{equation} \\ \\


\begin{equation}
    \begin{split}
        \begin{bmatrix}
            1\ 1\\
            2\ 2\\
        \end{bmatrix}
        \begin{bmatrix}
            1\ 1\\
            2\ 2\\
        \end{bmatrix}^{-1} = 
        \begin{bmatrix}
            1\ 0\\
            0\ 1\\
        \end{bmatrix} \\ \\
        \textbf{\textit{A}}\textbf{\textit{A}}^{-1} = \textbf{\textit{I}}
    \end{split}
\end{equation} \\ \\ 




$\textbf{Hadamard product}$ \\ \\ 


\begin{equation}
    \begin{split}
        \begin{bmatrix}
            1\ 1\\
            2\ 2\\
        \end{bmatrix}
        \odot
        \begin{bmatrix}
            1\ 2\\
            3\ 4\\
        \end{bmatrix} = 
        \begin{bmatrix}
            1\ 2\\
            6\ 8\\
        \end{bmatrix} \\ \\
        \textbf{\textit{A}}\odot\textbf{\textit{B}} = \textbf{\textit{C}} 
    \end{split}
\end{equation} \\ \\ 



$\textbf{Dot product}$ \\


when 
$\textbf{\textit{a}} = 
\begin{bmatrix} 
    1\\ 
    2\\ 
\end{bmatrix}$ 
and $\textbf{\textit{b}} = 
\begin{bmatrix} 
    3\\ 
    4\\ 
\end{bmatrix}$ 

\begin{equation}
    \textbf{\textit{a}} \cdot \textbf{\textit{b}} = 
    \textbf{\textit{a}}^T\textbf{\textit{b}} = 
    \begin{bmatrix}
        1\ 2\\
    \end{bmatrix}
    \begin{bmatrix}
        3\\
        4\\
    \end{bmatrix} = 11 
\end{equation} \\ \\


\pagebreak


\textbf{linear equation}


\begin{equation}
    \textbf{\textit{A}}\textbf{\textit{x}} = \textbf{\textit{b}}
\end{equation}

\begin{equation}
    \begin{bmatrix}
        a_0\ a_1\\
        a_2\ a_3\\
    \end{bmatrix}
    \begin{bmatrix}
        x_0\\
        x_1\\
    \end{bmatrix} = 
    \begin{bmatrix}
        b_0\\
        b_1\\
    \end{bmatrix}
\end{equation}


\begin{equation}
    \begin{split}
        b_0 = a_0x_0 + a_1x_1 \\
        b_1 = a_2x_0 + a_3x_1
    \end{split}
\end{equation} \\



\textbf{Linear Dependence and Span} 

\begin{equation}
    equation
\end{equation}


\textbf{Norm} \\ \\

Definition \\
\begin{equation}
    \Vert \textit{\textbf{x}}_{\textit{p}} \Vert = {\Big(\sum_{i}\vert\textit{x}_{i}\vert^{\textit{p}}\Big)}^{\frac{1}{\textit{p}}}
\end{equation}

\begin{equation}
    \begin{split}
        \textit{L}^{\textit{p}} = \Vert \textbf{\textit{x}} \Vert  \\
    \end{split}
\end{equation}


L1 norm \\ 
\begin{equation}
    \begin{split}
        L^1 = \sum_{i}\vert\textit{x}_{i}\vert \\ 
        = \vert x_0 \vert + \vert x_1\vert + \vert x_2\vert \dots \vert x_n \vert \\
    \end{split}
\end{equation}


L2 norm \\
\begin{equation}
    \begin{split}
        L^2 = {\Big(\sum_{i}\vert\textit{x}_{i}\vert^{2}\Big)}^{\frac{1}{2}} \\ 
        = \sqrt{x_0^2 + x_1^2 + x_2^2 \dots x_n^2} \\ 
    \end{split}
\end{equation}


Squared L2 norm \\ 
\begin{equation}
    \begin{split}
        squared \textit{L}^2 = \textbf{\textit{x}}^{T}\textbf{\textit{x}} \\ 
        = x_0^2 + x_1^2 + x_2^2 \dots x_n^2 \\ 
    \end{split}
\end{equation}

Frobenius norm \\
\begin{equation}
    \begin{split}
        {\Vert\textbf{\textit{A}}\Vert}_{\textit{F}} = \sqrt{\sum_{i,j}\textit{\textbf{A}}_{i,j}^{2}} \\ 
    \end{split}
\end{equation} \\ \\


\textbf{Norm} \\ \\



\end{document}
